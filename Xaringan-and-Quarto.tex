% Options for packages loaded elsewhere
\PassOptionsToPackage{unicode}{hyperref}
\PassOptionsToPackage{hyphens}{url}
\PassOptionsToPackage{dvipsnames,svgnames,x11names}{xcolor}
%
\documentclass[
  letterpaper,
  DIV=11,
  numbers=noendperiod]{scrreprt}

\usepackage{amsmath,amssymb}
\usepackage{iftex}
\ifPDFTeX
  \usepackage[T1]{fontenc}
  \usepackage[utf8]{inputenc}
  \usepackage{textcomp} % provide euro and other symbols
\else % if luatex or xetex
  \usepackage{unicode-math}
  \defaultfontfeatures{Scale=MatchLowercase}
  \defaultfontfeatures[\rmfamily]{Ligatures=TeX,Scale=1}
\fi
\usepackage{lmodern}
\ifPDFTeX\else  
    % xetex/luatex font selection
\fi
% Use upquote if available, for straight quotes in verbatim environments
\IfFileExists{upquote.sty}{\usepackage{upquote}}{}
\IfFileExists{microtype.sty}{% use microtype if available
  \usepackage[]{microtype}
  \UseMicrotypeSet[protrusion]{basicmath} % disable protrusion for tt fonts
}{}
\makeatletter
\@ifundefined{KOMAClassName}{% if non-KOMA class
  \IfFileExists{parskip.sty}{%
    \usepackage{parskip}
  }{% else
    \setlength{\parindent}{0pt}
    \setlength{\parskip}{6pt plus 2pt minus 1pt}}
}{% if KOMA class
  \KOMAoptions{parskip=half}}
\makeatother
\usepackage{xcolor}
\setlength{\emergencystretch}{3em} % prevent overfull lines
\setcounter{secnumdepth}{-\maxdimen} % remove section numbering
% Make \paragraph and \subparagraph free-standing
\ifx\paragraph\undefined\else
  \let\oldparagraph\paragraph
  \renewcommand{\paragraph}[1]{\oldparagraph{#1}\mbox{}}
\fi
\ifx\subparagraph\undefined\else
  \let\oldsubparagraph\subparagraph
  \renewcommand{\subparagraph}[1]{\oldsubparagraph{#1}\mbox{}}
\fi


\providecommand{\tightlist}{%
  \setlength{\itemsep}{0pt}\setlength{\parskip}{0pt}}\usepackage{longtable,booktabs,array}
\usepackage{calc} % for calculating minipage widths
% Correct order of tables after \paragraph or \subparagraph
\usepackage{etoolbox}
\makeatletter
\patchcmd\longtable{\par}{\if@noskipsec\mbox{}\fi\par}{}{}
\makeatother
% Allow footnotes in longtable head/foot
\IfFileExists{footnotehyper.sty}{\usepackage{footnotehyper}}{\usepackage{footnote}}
\makesavenoteenv{longtable}
\usepackage{graphicx}
\makeatletter
\def\maxwidth{\ifdim\Gin@nat@width>\linewidth\linewidth\else\Gin@nat@width\fi}
\def\maxheight{\ifdim\Gin@nat@height>\textheight\textheight\else\Gin@nat@height\fi}
\makeatother
% Scale images if necessary, so that they will not overflow the page
% margins by default, and it is still possible to overwrite the defaults
% using explicit options in \includegraphics[width, height, ...]{}
\setkeys{Gin}{width=\maxwidth,height=\maxheight,keepaspectratio}
% Set default figure placement to htbp
\makeatletter
\def\fps@figure{htbp}
\makeatother

\KOMAoption{captions}{tableheading}
\makeatletter
\@ifpackageloaded{caption}{}{\usepackage{caption}}
\AtBeginDocument{%
\ifdefined\contentsname
  \renewcommand*\contentsname{Table of contents}
\else
  \newcommand\contentsname{Table of contents}
\fi
\ifdefined\listfigurename
  \renewcommand*\listfigurename{List of Figures}
\else
  \newcommand\listfigurename{List of Figures}
\fi
\ifdefined\listtablename
  \renewcommand*\listtablename{List of Tables}
\else
  \newcommand\listtablename{List of Tables}
\fi
\ifdefined\figurename
  \renewcommand*\figurename{Figure}
\else
  \newcommand\figurename{Figure}
\fi
\ifdefined\tablename
  \renewcommand*\tablename{Table}
\else
  \newcommand\tablename{Table}
\fi
}
\@ifpackageloaded{float}{}{\usepackage{float}}
\floatstyle{ruled}
\@ifundefined{c@chapter}{\newfloat{codelisting}{h}{lop}}{\newfloat{codelisting}{h}{lop}[chapter]}
\floatname{codelisting}{Listing}
\newcommand*\listoflistings{\listof{codelisting}{List of Listings}}
\makeatother
\makeatletter
\makeatother
\makeatletter
\@ifpackageloaded{caption}{}{\usepackage{caption}}
\@ifpackageloaded{subcaption}{}{\usepackage{subcaption}}
\makeatother
\ifLuaTeX
  \usepackage{selnolig}  % disable illegal ligatures
\fi
\usepackage{bookmark}

\IfFileExists{xurl.sty}{\usepackage{xurl}}{} % add URL line breaks if available
\urlstyle{same} % disable monospaced font for URLs
\hypersetup{
  pdftitle={Week 2 Protfolio},
  colorlinks=true,
  linkcolor={blue},
  filecolor={Maroon},
  citecolor={Blue},
  urlcolor={Blue},
  pdfcreator={LaTeX via pandoc}}

\title{Week 2 Protfolio}
\author{}
\date{}

\begin{document}
\maketitle

\section{Summary}\label{summary}

Xaringan is a new way of doing Slides, much simpler and more
straightforward, implemented in R Markdown. Xaringan overrides all the
functionality of regular Slides and provides some additional
functionality. For example: directly implant code, forms, etc

I'll show you my use of Xaringan in the next section(3.2)

Quarto itself supports building slideshows similar to Xaringan. Below
here I will show the Slide with the image and code that is not clearly
shown above.

\section{Reflection}\label{reflection}

The things we learnt this week are very effective in academic research
and technical presentations; Xaringan can be used to make clear and
concise academic presentations and Quarto is good for writing technical
documentation or online books. With GitHub Pages, these resources can be
easily shared with others.

In the literature, many studies have used Xaringan and Quarto to present
data analysis results or technical methods. For example, some studies
have used Quarto to write data analysis reports and then publish them as
online books that readers can browse interactively.

However, there are some limitations to these tools, such as lack of
support for complex formats and the need for some programming skills to
use them well.

\textbf{💡:}

Quarto's support for multiple languages feels like it would be useful in
interdisciplinary research, but in practice, how do you make sure that
blocks of code in different languages fit together seamlessly and
display correctly? For example, combining data analysis in R with
machine learning models in Python.

(Possibly via the Jupyter Kernel, or some package that supports
interconnections (reticulate)?)

\section{Application}\label{application}

\subsection{Use of Xaringan}\label{use-of-xaringan}

In Slide, I recorded what I learned about Xaringan this week, as well as
some of my own development. However, the direct display of Slide does
not seem to show the results of images and code blocks directly.

\subsection{Use of Quarto}\label{use-of-quarto}

\begin{center}\rule{0.5\linewidth}{0.5pt}\end{center}

\textbf{Import pictures and table \textasciitilde{}}

\includegraphics[width=1\textwidth,height=\textheight]{week2/coding.jpeg}

This is the description of the image, aligned to the right of the
picture.

\textbf{View the head of iris data} (in bold)

\begin{longtable}[]{@{}rrrrl@{}}
\toprule\noalign{}
Sepal.Length & Sepal.Width & Petal.Length & Petal.Width & Species \\
\midrule\noalign{}
\endhead
\bottomrule\noalign{}
\endlastfoot
5.1 & 3.5 & 1.4 & 0.2 & setosa \\
4.9 & 3.0 & 1.4 & 0.2 & setosa \\
4.7 & 3.2 & 1.3 & 0.2 & setosa \\
4.6 & 3.1 & 1.5 & 0.2 & setosa \\
5.0 & 3.6 & 1.4 & 0.2 & setosa \\
5.4 & 3.9 & 1.7 & 0.4 & setosa \\
\end{longtable}

\begin{center}\rule{0.5\linewidth}{0.5pt}\end{center}



\end{document}
